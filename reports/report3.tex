\documentclass[12pt,a4paper]{article}
\usepackage[legalpaper, portrait, margin=2cm]{geometry}
\usepackage{fancyhdr}
\usepackage{amsmath}
\usepackage{amssymb}
\usepackage{graphicx}
\usepackage{wrapfig}
\usepackage{blindtext}
\usepackage{hyperref}
\usepackage{enumitem}
\usepackage{pdflscape}
\usepackage{svg}
\usepackage{listings}
\usepackage{xcolor}

\graphicspath{ {./} }
\hypersetup{
  colorlinks=true,
  linkcolor=blue,
  filecolor=magenta,
  urlcolor=blue,
  citecolor=blue,
  pdftitle={Entrega 3 - Relatório BD 2022/2023 LEIC-A},
  pdfpagemode=FullScreen,
}

\pagestyle{fancy}
\fancyhf{}
\rhead{Grupo \textbf{2}}
\lhead{Entrega 3 Relatório BD 2022/2023 LEIC-A}
\cfoot{Gonçalo Bárias (103124), Raquel Braunschweig (102624) e Vasco Paisana (102533)}

\renewcommand{\footrulewidth}{0.2pt}

\renewcommand{\labelitemii}{$\circ$}
\renewcommand{\labelitemiii}{$\diamond$}
\newcommand{\op}{\text}

\newlist{constraintsList}{itemize}{4}
\setlist[constraintsList]{itemsep=1pt, topsep=1pt, label=\protect\mpbullet}

\definecolor{codegreen}{rgb}{0,0.6,0}
\definecolor{codegray}{rgb}{0.5,0.5,0.5}
\definecolor{codepurple}{rgb}{0.58,0,0.82}
\definecolor{backcolour}{rgb}{0.95,0.95,0.92}

\lstdefinestyle{mystyle}{
    commentstyle=\color{codegreen},
    keywordstyle=\color{magenta},
    numberstyle=\tiny\color{codegray},
    stringstyle=\color{codepurple},
    basicstyle=\ttfamily\footnotesize,
    breakatwhitespace=false,
    breaklines=true,
    captionpos=b,
    keepspaces=true,
    numbers=left,
    numbersep=5pt,
    showspaces=false,
    showstringspaces=false,
    showtabs=false,
    tabsize=2
}

\lstset{style=mystyle}
% https://stackoverflow.com/questions/1116266/listings-in-latex-with-utf-8-or-at-least-german-umlauts
\lstset{%
        inputencoding=utf8,
        extendedchars=true,
        literate=%
        {é}{{\'{e}}}1
        {è}{{\`{e}}}1
        {ê}{{\^{e}}}1
        {ë}{{\¨{e}}}1
        {É}{{\'{E}}}1
        {Ê}{{\^{E}}}1
        {û}{{\^{u}}}1
        {ù}{{\`{u}}}1
        {ú}{{\'{u}}}1
        {â}{{\^{a}}}1
        {à}{{\`{a}}}1
        {á}{{\'{a}}}1
        {ã}{{\~{a}}}1
        {Á}{{\'{A}}}1
        {Â}{{\^{A}}}1
        {Ã}{{\~{A}}}1
        {ç}{{\c{c}}}1
        {Ç}{{\c{C}}}1
        {õ}{{\~{o}}}1
        {ó}{{\'{o}}}1
        {ô}{{\^{o}}}1
        {Õ}{{\~{O}}}1
        {Ó}{{\'{O}}}1
        {Ô}{{\^{O}}}1
        {î}{{\^{i}}}1
        {Î}{{\^{I}}}1
        {í}{{\'{i}}}1
        {Í}{{\~{Í}}}1
}

\begin{document}
\begin{titlepage}
  \begin{center}
    \vspace*{5cm}
    \Huge
    \textbf{Projeto de BD - Parte 3}

    \vspace{0.5cm}
    \LARGE
    Grupo 002 | Turno L13 | LEIC-A

    \vspace{0.5cm}
    \large
    Prof. Flávio Martins

    \vspace{0.5cm}
    \begin{figure}[h]
      \centering
      \includegraphics[scale=0.5]{report_logo.png}
    \end{figure}

    \vfill
    \large
    \begin{minipage}{0.8\textwidth}
      \begin{itemize}
        \item[] \textbf{Gonçalo Bárias} (103124) - 33.3$\overline{3}$\% - 36h
        \item[] \textbf{Raquel Braunschweig} (102624) - 33.3$\overline{3}$\% - 36h
        \item[] \textbf{Vasco Paisana} (102533) - 33.3$\overline{3}$\% - 36h
      \end{itemize}
    \end{minipage}
  \end{center}
\end{titlepage}

\section*{Desenvolvimento da Aplicação}

\section*{Índices}

Foi pedido, ainda, que se indicasse (justificando) os índices que faria sentido
criar, por forma a \textit{agilizar} a execução de cada uma das \textit{queries}
apresentadas de seguida.

\subsection*{Primeira \textit{query}:}

\lstinputlisting[language=SQL]{sql/index_query1.sql}

Para a primeira \textit{query}, optou-se por criar dois índices: um no atributo
\texttt{price} da relação \texttt{product}, e outro no ano do atributo \texttt{date}
da relação \texttt{orders}.

\vspace*{0.25cm}

\begin{itemize}
  \item Não foi criado nenhum índice para os atributos \texttt{order\_no} e \texttt{SKU} de \texttt{contains}, pois o
        \texttt{PostgreSQL} já cria um índice \textbf{BTree} para (\texttt{order\_no}, \texttt{SKU}) dado que se trata
        de uma chave primária.
        O \texttt{planner} é capaz de usar este índice e assim não se justifica criar dois
        índices separados para estes atributos apenas devido às clásulas \texttt{JOIN}.

  \item Optámos pela criação de um índice \textbf{BTree} no atributo \texttt{price} de \texttt{product}, pois
        a comparação pretendida engloba um intervalo de preços e assim um índice \textbf{Hash}
        não seria particularmente inteligente, já que não se pretende um único preço
        em concreto.

  \item Por fim, criámos um índice para o ano do atributo \texttt{date} de \texttt{orders}, pois
        estando a fazer uma comparação de igualdade no ano das datas das encomendas,
        faz todo o sentido usar um índice \textbf{Hash} já que a comparação em $O(1)$ é ideal.
\end{itemize}

O trecho de código correspondente à indexação pretendida encontra-se abaixo:

\lstinputlisting[language=SQL]{sql/index1.sql}

\subsection*{Segunda \textit{query}:}

\lstinputlisting[language=SQL]{sql/index_query2.sql}

Para a segunda \textit{query}, optou-se por criar um índice no atributo \texttt{name} da relação
\texttt{product}.

\vspace*{0.25cm}

\begin{itemize}
  \item Aqui a clásula \texttt{GROUP BY} beneficia de um índice \textbf{BTree} em \texttt{order\_no} de
        \texttt{contains}, pois o \texttt{GROUP BY} procura agrupar os dados sobre o atributo \texttt{order\_no}
        e os índices \textbf{BTree} já vêm, por natureza, ordenados.
        Porém, tal como no ponto anterior, o \texttt{planner} é capaz de usar o índice criado
        na chave primária de \texttt{contains} para agilizar a computação do \texttt{GROUP BY}, sendo
        esse já um índice \textbf{BTree}.

  \item Assim, apenas é necessário criar um índice \textbf{BTree} para o atributo \texttt{name}
        de \texttt{product}, pois a comparação pretendida engloba todo um intervalo de nomes
        de produtos que começam pela letra \texttt{A}, logo um índice \textbf{Hash} não ajudaria.
\end{itemize}

O trecho de código correspondente à indexação pretendida encontra-se abaixo:

\lstinputlisting[language=SQL]{sql/index2.sql}

\end{document}
