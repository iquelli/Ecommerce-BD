\documentclass[12pt, a4paper]{article}
\usepackage[legalpaper, portrait, margin=3cm]{geometry}
\usepackage{fancyhdr}
\usepackage{amsmath}
\usepackage{amssymb}
\usepackage{graphicx}
\usepackage{wrapfig}
\usepackage{blindtext}
\usepackage{hyperref}
\usepackage{pdflscape}
\usepackage{svg}

\graphicspath{ {./} }
\hypersetup{
  colorlinks=true,
  linkcolor=blue,
  filecolor=magenta,
  urlcolor=blue,
  citecolor=blue,
  pdftitle={Entrega 1 - Relatório BD 2022/2023 LEIC-A},
  pdfpagemode=FullScreen,
}

\pagestyle{fancy}
\fancyhf{}
\rhead{Grupo \textbf{2}}
\lhead{Entrega 1 Relatório BD 2022/2023 LEIC-A}
\cfoot{Gonçalo Bárias (103124), Raquel Braunschweig (102624) e Vasco Paisana (102533)}

\renewcommand{\footrulewidth}{0.2pt}

\renewcommand{\labelitemii}{$\circ$}
\renewcommand{\labelitemiii}{$\diamond$}

\begin{document}
  \begin{titlepage}
    \begin{center}
      \vspace*{5cm}
      \Huge
      \textbf{Projeto BD - Parte 1}

      \vspace{0.5cm}
      \LARGE
      Grupo 002 | Turno L13 | LEIC-A

      \vspace{0.5cm}
      \large
      Prof. Flávio Martins

      \vspace{0.5cm}
      \begin{figure}[h]
          \centering
          \includegraphics[scale=0.5]{report_logo.png}
      \end{figure}

      \vfill
      \large
      \begin{minipage}{0.8\textwidth}
        \begin{itemize}
          \item[] \textbf{Gonçalo Bárias} (103124) - 33.3$\overline{3}$\% - 10h
          \item[] \textbf{Raquel Braunschweig} (102624) - 33.3$\overline{3}$\% - 10h
          \item[] \textbf{Vasco Paisana} (102533) - 33.3$\overline{3}$\% - 10h
        \end{itemize}
      \end{minipage}
    \end{center}
  \end{titlepage}

  \begin{landscape}
    \centering
    \includesvg[width=1.79\textwidth]{domain_model}
  \end{landscape}

  \section*{Restrições de Integridade}
  \footnotesize
  \begin{itemize}
    \item[\textbf{(RI-1)}] O email do cliente é \textbf{único}.
    \item[\textbf{(RI-2)}] Cada método de pagamento \textbf{tem de ter um método diferente} como substituto em caso de falha.
    \item[\textbf{(RI-3)}] Cada venda é \textbf{sempre} registada com o método de pagamento \textbf{bem sucedido} do cliente que efetuou a encomenda.
    \item[\textbf{(RI-4)}] Um empregado \textbf{só pode} processar encomendas que contenham produtos do mesmo armazém onde ele trabalha.
  \end{itemize}

  \section*{Justificações de Desenho}
  \normalsize
  \begin{itemize}
    \item Modelámos o departamento como uma entidade. Como nada no enunciado proíbe o empregado de pertencer a mais do que um departamento não o podemos
    representar como um atributo. Assim, criámos uma chave primária (\textbf{dep\_id}), tornando-o uma entidade forte.
    \item Achámos redundante associar diretamente o cliente a uma encomenda para o evento de ele a pagar, pois já existe uma associação entre
    venda e método de pagamento, sendo este último uma entidade fraca do cliente. Isto, com a restrição de integridade número 3 (\textbf{RI-3}),
    já relaciona o cliente com a venda através do seu método de pagamento com que efetuou corretamente o pagamento.
    \item Interpretámos \textit{"Neste sistema os clientes têm de introduzir mais do que um método de pagamento e cada método de pagamento
    tem de ter outro método como substituto em caso de falha."} como sendo apenas necessário que cada cliente tenha um método de pagamento principal.
    Como o método principal tem outro método como substituto, o requisito de cada cliente introduzir mais do que um método fica assim satisfeito.
    \item Decidimos criar uma entidade para a morada (\textbf{Address}) para poder discriminar entre os seus campos e para garantir
    que cada local de trabalho (\textbf{Workplace}) tem uma morada única sem ser necessário utilizar uma restrição de integridade.
    \item Considerámos que a frase \textit{"Devido a restrições de privacidade, os operadores do sistema não podem ver os nomes dos clientes."}
    não constitui um requisito de desenho, pois é uma funcionalidade que não tem manifestação ao nível dos dados, pelo que não foi incorporada no diagrama.
    \item Pela mesma razão, decidimos não incluir nada sobre a imutabilidade do número de cliente e do número da encomenda no diagrama, nem nada
    sobre a frase \textit{"Os métodos de pagamento são escolhidos de uma lista conhecida à priori."}
  \end{itemize}
\end{document}
