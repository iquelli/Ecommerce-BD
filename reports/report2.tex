\documentclass[12pt,a4paper]{article}
\usepackage[legalpaper, portrait, margin=2cm]{geometry}
\usepackage{fancyhdr}
\usepackage{amsmath}
\usepackage{amssymb}
\usepackage{graphicx}
\usepackage{wrapfig}
\usepackage{blindtext}
\usepackage{hyperref}
\usepackage{enumitem}
\usepackage{pdflscape}
\usepackage{svg}
\usepackage{listings}
\usepackage{xcolor}

\graphicspath{ {./} }
\hypersetup{
    colorlinks=true,
    linkcolor=blue,
    filecolor=magenta,
    urlcolor=blue,
    citecolor=blue,
    pdftitle={Entrega 2 - Relatório BD 2022/2023 LEIC-A},
    pdfpagemode=FullScreen,
}

\pagestyle{fancy}
\fancyhf{}
\rhead{Grupo \textbf{2}}
\lhead{Entrega 2 Relatório BD 2022/2023 LEIC-A}
\cfoot{Gonçalo Bárias (103124), Raquel Braunschweig (102624) e Vasco Paisana (102533)}

\renewcommand{\footrulewidth}{0.2pt}

\renewcommand{\labelitemii}{$\circ$}
\renewcommand{\labelitemiii}{$\diamond$}
\newcommand{\op}{\text}

\newlist{constraintsList}{itemize}{4}
\setlist[constraintsList]{itemsep=1pt, topsep=1pt, label=\protect\mpbullet}

\definecolor{codegreen}{rgb}{0,0.6,0}
\definecolor{codegray}{rgb}{0.5,0.5,0.5}
\definecolor{codepurple}{rgb}{0.58,0,0.82}
\definecolor{backcolour}{rgb}{0.95,0.95,0.92}

\lstdefinestyle{mystyle}{
    commentstyle=\color{codegreen},
    keywordstyle=\color{magenta},
    numberstyle=\tiny\color{codegray},
    stringstyle=\color{codepurple},
    basicstyle=\ttfamily\footnotesize,
    breakatwhitespace=false,
    breaklines=true,
    captionpos=b,
    keepspaces=true,
    numbers=left,
    numbersep=5pt,
    showspaces=false,
    showstringspaces=false,
    showtabs=false,
    tabsize=2
}

\lstset{style=mystyle}
% https://stackoverflow.com/questions/1116266/listings-in-latex-with-utf-8-or-at-least-german-umlauts
\lstset{%
inputencoding=utf8,
extendedchars=true,
literate=%
{é}{{\'{e}}}1
{è}{{\`{e}}}1
{ê}{{\^{e}}}1
{ë}{{\¨{e}}}1
{É}{{\'{E}}}1
{Ê}{{\^{E}}}1
{û}{{\^{u}}}1
{ù}{{\`{u}}}1
{ú}{{\'{u}}}1
{â}{{\^{a}}}1
{à}{{\`{a}}}1
{á}{{\'{a}}}1
{ã}{{\~{a}}}1
{Á}{{\'{A}}}1
{Â}{{\^{A}}}1
{Ã}{{\~{A}}}1
{ç}{{\c{c}}}1
{Ç}{{\c{C}}}1
{õ}{{\~{o}}}1
{ó}{{\'{o}}}1
{ô}{{\^{o}}}1
{Õ}{{\~{O}}}1
{Ó}{{\'{O}}}1
{Ô}{{\^{O}}}1
{î}{{\^{i}}}1
{Î}{{\^{I}}}1
{í}{{\'{i}}}1
{Í}{{\~{Í}}}1
}

\begin{document}
  \begin{titlepage}
    \begin{center}
      \vspace*{5cm}
      \Huge
      \textbf{Projeto de BD - Parte 2}

      \vspace{0.5cm}
      \LARGE
      Grupo 002 | Turno L13 | LEIC-A

      \vspace{0.5cm}
      \large
      Prof. Flávio Martins

      \vspace{0.5cm}
      \begin{figure}[h]
        \centering
        \includegraphics[scale=0.5]{report_logo.png}
      \end{figure}

      \vfill
      \large
      \begin{minipage}{0.8\textwidth}
        \begin{itemize}
          \item[] \textbf{Gonçalo Bárias} (103124) - 33.3$\overline{3}$\% - 12h
          \item[] \textbf{Raquel Braunschweig} (102624) - 33.3$\overline{3}$\% - 12h
          \item[] \textbf{Vasco Paisana} (102533) - 33.3$\overline{3}$\% - 12h
        \end{itemize}
      \end{minipage}
    \end{center}
  \end{titlepage}

  \section*{Modelo Relacional}

  \ttfamily
  \noindent
  customer(\underline{cust\_no}, name, email, phone, address)
  \begin{itemize}[nosep]
      \item UNIQUE(email)
  \end{itemize}

  \vspace*{10pt}
  \noindent
  package(\underline{package\_no}, date, cust\_no)
  \begin{itemize}[nosep]
      \item cust\_no: FK(customer) NOT NULL
      \item \textsf{\textbf{(IC-6)}}: any package\_no in package \textsf{must exist} in contains
      \item \textsf{\textbf{(IC-7)}}: when a package \textsf{is removed} from the database it \textsf{must also} be removed from sale if present
  \end{itemize}

  \vspace*{10pt}
  \noindent
  sale(\underline{package\_no})
  \begin{itemize}[nosep]
      \item package\_no: FK(package)
  \end{itemize}

  \vspace*{10pt}
  \noindent
  pay(\underline{package\_no}, cust\_no)
  \begin{itemize}[nosep]
      \item package\_no: FK(sale)
      \item cust\_no: FK(customer) NOT NULL
      \item \textsf{\textbf{(IC-1)}}: cust\_no \textsf{must exist} in the package identified by package\_no
  \end{itemize}

  \vspace*{10pt}
  \noindent
  product(\underline{sku}, name, description, price)
  \begin{itemize}[nosep]
      \item \textsf{\textbf{(IC-8)}}: any sku in product \textsf{must exist} in supplier
      \item \textsf{\textbf{(IC-9)}}: when a product \textsf{is removed} from the database it \textsf{must also} be removed from ean\_product if present
  \end{itemize}

  \vspace*{10pt}
  \noindent
  ean\_product(\underline{sku}, ean)
  \begin{itemize}[nosep]
      \item sku: FK(product)
  \end{itemize}

  \vspace*{10pt}
  \noindent
  contains(\underline{package\_no},\underline{sku}, qty)
  \begin{itemize}[nosep]
      \item package\_no: FK(package)
      \item sku: FK(product)
  \end{itemize}

  \vspace*{10pt}
  \noindent
  supplier(\underline{tin}, name, address, sku, date)
  \begin{itemize}[nosep]
      \item sku: FK(product) NOT NULL
  \end{itemize}

  \vspace*{10pt}
  \noindent
  department(\underline{name})

  \vspace*{10pt}
  \noindent
  workplace(\underline{address}, lat, long)
  \begin{itemize}[nosep]
      \item UNIQUE(lat, long)
      \item \textsf{\textbf{(IC-10)}}: when a workplace \textsf{is removed} from the database it \textsf{must also} be removed from warehouse \textsf{and/or} office if present
  \end{itemize}

  \vspace*{10pt}
  \noindent
  warehouse(\underline{address})
  \begin{itemize}[nosep]
      \item address: FK(workplace)
  \end{itemize}

  \vspace*{10pt}
  \noindent
  delivery(\underline{address}, \underline{tin})
  \begin{itemize}[nosep]
      \item address: FK(warehouse)
      \item tin: FK(supplier)
  \end{itemize}

  \vspace*{10pt}
  \noindent
  office(\underline{address})
  \begin{itemize}[nosep]
      \item address: FK(workplace)
  \end{itemize}

  \vspace*{10pt}
  \noindent
  employee(\underline{ssn}, tin, b\_date, name)
  \begin{itemize}[nosep]
      \item UNIQUE(tin)
      \item \textsf{\textbf{(IC-11)}}: any ssn in employee \textsf{must exist} in works
  \end{itemize}

  \vspace*{10pt}
  \noindent
  works(\underline{ssn}, \underline{name}, \underline{address})
  \begin{itemize}[nosep]
      \item ssn: FK(employee)
      \item name: FK(department)
      \item address: FK(workplace)
  \end{itemize}

  \vspace*{10pt}
  \noindent
  process(\underline{ssn}, \underline{package\_no})
  \begin{itemize}[nosep]
      \item ssn: FK(employee)
      \item package\_no: FK(package)
  \end{itemize}
  \sffamily

  \vspace*{10pt}
  \noindent
  Renomeámos a relação \texttt{order} para \texttt{package} na conversão para o modelo relacional, pois em \texttt{PostgreSQL} a palavra \texttt{order} é uma keyword protegida.
  Mantivemos a numeração das \textit{Integrity Constraints} consoante as fornecidas no enunciado, iniciando a contagem a 6 para as \textit{ICs} adicionadas.
  A única \textit{IC} que não foi passível de converter para o modelo relacional foi a \textbf{(IC-1)}, sendo todas as outras convertidas através da propriedade \texttt{UNIQUE}.

  \section*{Álgebra Relacional}

  \begin{enumerate}
    \item \textbf{Liste o nome de todos os clientes que fizeram encomendas contendo produtos de preço superior a 50\textmd{\texteuro} no ano de 2023.}
    \[
      \begin{aligned}
        & c \; \leftarrow \; \sigma_{\op{date} \; \geq \; \text{'2023/01/01'} ~\land~ \op{date} \; \leq \; \text{'2023/12/31'} \;}(\op{package}) \bowtie \op{customer} \bowtie \op{contains} \\
        & \Pi _{\; \op{customer.name} \;}( \sigma _{\op{price} \; > \; \op{50} \;} (c \bowtie _{\; \op{contains.sku = product.sku}}\op{product}))
      \end{aligned}
    \]

    \item \textbf{Liste o nome de todos os empregados que trabalham em armazéns e não em escritórios e processaram encomendas em Janeiro de 2023.}
    \[
      \begin{aligned}
        & e \; \leftarrow \; \sigma_{\op{date} \; \geq \; \text{'2023/01/01'} ~\land~ \op{date} \; \leq \; \text{'2023/01/31'} \;}(\op{package}) \bowtie \op{process} \bowtie \op{employee} \bowtie _{\; \op{employee.ssn = works.ssn}} \op{works} \\
        & \Pi _{\; \op{employee.name} \;}((e \bowtie \op{warehouse}) - (e \bowtie \op{office}))
      \end{aligned}
    \]

    \item \textbf{Indique o nome do produto mais vendido.}

    Em caso de empate, apresentamos todos os produtos mais vendidos.
    \[
      \begin{aligned}
        & p \; \leftarrow \; _{\op{sku} \;} G _{\; \op{sum(qty)} \; \mapsto \; \op{p\_qty} \;}(\op{contains} \bowtie \op{sale}) \\
        & \Pi _{\; \op{name} \;}(G _{\; \op{max(p\_qty)} \; \mapsto \; \op{p\_qty} \;}(p) \bowtie p \bowtie \op{product})
      \end{aligned}
    \]

    \item \textbf{Indique o valor total de cada venda realizada.}

    Apresentámos junto do valor total o correspondente \texttt{package\_no} da venda, pois para casos em que
    há duas vendas com o mesmo valor total não seria possível de as distinguir.
    \[
      \begin{aligned}
          & _{\op{package\_no} \;} G _{\; \op{sum}(\op{p\_val}) \; \mapsto \; \op{total\_val} \;}(\rho _{\; \op{r} (\op{3} \; \mapsto \; \op{p\_val}) \;}(\Pi _{ \; \op{package\_no}, \; \op{sku}, \; \op{price} \; * \; \op{qty} \;}(\op{sale} \bowtie \op{contains} \bowtie \op{product})))
      \end{aligned}
    \]
  \end{enumerate}
\end{document}
